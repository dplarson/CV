\documentclass[]{res}

% ignore underfull warnings
\hbadness=10000

% increase text height to fit on 1-page
\setlength{\textheight}{9.5in}

\usepackage{hyperref}


\begin{document}

\address{\textbf{David P. Larson} \\
    9500 Gilman Drive \#0411 \\
    La Jolla, CA 92093--0411
}
\address{\hfill \href{mailto:dplarson@ucsd.edu}{dplarson@ucsd.edu} \\
    \hfill \url{http://ieng6.ucsd.edu/~dplarson} \\
    \hfill \url{http://github.com/dplarson}
}

\begin{resume}

%-----------------------------------------------------------------------------
% OBJECTIVE
%-----------------------------------------------------------------------------
%\section{Objective}
%A summer position focused on

%-----------------------------------------------------------------------------
% EDUCATION
%-----------------------------------------------------------------------------
\section{Education}
\vspace{0.1in}

\textbf{University of California, San Diego} \hfill in progress \\
Ph.D., Mechanical Engineering \\

\vspace{-0.2in}

\textbf{University of California, San Diego} \hfill 2014 \\
M.S., Mechanical Engineering \\

\vspace{-0.2in}

\textbf{University of California, Merced} \hfill 2012 \\
B.S., Mechanical Engineering


%-----------------------------------------------------------------------------
% Experience
%-----------------------------------------------------------------------------
\section{Experience}
\vspace{0.1in}

% Lab Manager
\textbf{Lab Manager}, Coimbra Energy Group \hfill 2012--present \\
University of California, San Diego \\
Supervise lab usage of $\sim$15 graduate and undergraduate student researchers. \\
Train lab members on proper and safe usage of research equipment.

% GSR for CEC project
\textbf{Graduate Student Researcher}, Coimbra Energy Group \hfill 2012--present \\
University of California, San Diego \\
Developing machine learning-based forecast models of power output from solar plants. \\
Leading the development of a distributed, wireless monitoring system for solar power plants. \\
Automated the data collection of remote sensors across California and Washington.

\textbf{Teaching Assistant}, Intro to Engineering (ENG 1,2,3) \hfill 2013--present \\
University of California, San Diego \\
Developed and presented weekly lectures to 30--50 freshmen engineering students.



%-----------------------------------------------------------------------------
% SKILLS
%-----------------------------------------------------------------------------
\section{Skills}
\vspace{0.1in}

\textbf{Data Science:} machine learning, numerical optimization, data
visualization, statistical data analysis, image processing, time-series
analysis

\textbf{Software:} Python (NumPy, SciPy, Pandas, scikit-learn, iPython),
MATLAB, Mathematica, C, Go, Julia, SQL, shell scripting, Git, Microsoft Office,
LaTeX, Pro/ENGINEER, Solidworks, command-line tools (vim, ssh, etc.)

\textbf{Hardware:} Arduino, Beaglebone, Raspberry Pi, XBee, analog and digital
sensors, I$^2$C, SPI, UART, machining (mill, lathe, CNC)

\textbf{Platforms}: Mac OS X, Linux, Windows



%-----------------------------------------------------------------------------
% PUBLICATIONS
%-----------------------------------------------------------------------------
\section{Publications}
\vspace{0.1in}

D.P. Larson, L. Nonnenmacher and C.F.M. Coimbra (2016). \textbf{Day-Ahead Forecasting of Solar Power Output from Photovoltaic Plants in the American Southwest}, Renewable Energy (91), pp. 11--20.


%-----------------------------------------------------------------------------
% PROFESSIONAL
%-----------------------------------------------------------------------------
\section{Professional Activities}
\vspace{0.1in}

\textbf{Journal Reviewer:} Renewable Energy, Solar Energy, AMS JAMC

\textbf{Professional Societies:} ASME Student Member (2009--present)

%\textbf{Open Source Software:} Homebrew (contributor, 2012--present)


\end{resume}
\end{document}
