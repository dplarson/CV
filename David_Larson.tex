\documentclass[]{res}

% ignore underfull warnings
\hbadness=10000

% increase text height to fit on 1-page
\setlength{\textheight}{9.5in}

\usepackage{hyperref}


\begin{document}

\name{David P. Larson}
\address{University of California, San Diego \\ 9500 Gilman Drive \#0411 \\ La Jolla, CA 92093--0411}
\address{\hfill \href{mailto:dplarson@ucsd.edu}{dplarson@ucsd.edu} \\
    \hfill \url{http://ieng6.ucsd.edu/~dplarson} \\
    \hfill \url{http://github.com/dplarson}
}


\begin{resume}


%-----------------------------------------------------------------------------
% EDUCATION
%-----------------------------------------------------------------------------
\section{Education}
\vspace{0.1in}

\textbf{University of California, San Diego} \hfill in progress \\
Ph.D., Mechanical Engineering \\
Advisor: Carlos F.M. Coimbra \\

\vspace{-0.2in}

\textbf{University of California, San Diego} \hfill 2014 \\
M.S., Mechanical Engineering \\

\vspace{-0.2in}

\textbf{University of California, Merced} \hfill 2012 \\
B.S., Mechanical Engineering


%-----------------------------------------------------------------------------
% RESEARCH
%-----------------------------------------------------------------------------
\section{Research Experience}
\vspace{0.1in}

\textbf{University of California, San Diego}, with Carlos F.M. Coimbra \hfill 2012--present\\
Forecasting power output of solar power plants.

\textbf{University of California, Berkeley}, with Robert Dudley \hfill Summer 2011 \\
\textit{Visiting UC LEADS Scholar}, Cal NERDS Program \\
Evaluated the effects of turbulence on hummingbird flight dynamics.

\textbf{University of California, Merced}, with Carlos F.M. Coimbra \hfill Summer 2010 \\
\textit{UC LEADS Scholar} \\
Assisted in the development of an experiment for studying insect flight aeroelastics.


%-----------------------------------------------------------------------------
% WORK
%-----------------------------------------------------------------------------
%\section{Work Experience}
%\vspace{0.1in}
%% UC Merced: PIER-RESCO project
%% UCSD: GSR, TA
%%
%
%\textbf{UC Merced}, Undergraduate Researcher \hfill 2009--2012 \\
%PIER RESCO Solar Community at UC Merced


%-----------------------------------------------------------------------------
% PUBLICATIONS
%-----------------------------------------------------------------------------
\section{Journal Publications}
\vspace{0.1in}

D.P. Larson, L. Nonnenmacher and C.F.M. Coimbra (2016). \textbf{Day-ahead forecasting of solar power output from photovoltaic plants in the American Southwest}, Renewable Energy (91), pp. 11--20.


%-----------------------------------------------------------------------------
% TEACHING
%-----------------------------------------------------------------------------
%\vspace{-0.1in}
\section{Teaching Experience}
\vspace{0.1in}

\textbf{ENG 1: Orientation to Engineering I}, Teaching Assistant \hfill Fall 2013, 2014, 2015 \\
\textit{University of California, San Diego} \\
Lectured on academic planning, time management, and study habits.

\textbf{ENG 2: Orientation to Engineering II}, Teaching Assistant \hfill Winter 2014, 2015, 2016 \\
\textit{University of California, San Diego} \\
Lectured on career planning, business etiquette, r\'esum\'e development, and presentation skills.

\textbf{ENG 3: Orientation to Engineering III}, Teaching Assistant \hfill Spring 2014, 2015 \\
\textit{University of California, San Diego} \\
Lectured on project management, engineering as a profession, and ethics. \\
Developed interactive design activities.


%-----------------------------------------------------------------------------
% CONFERENCES
%-----------------------------------------------------------------------------
%\section{Conference Presentations}
%\vspace{0.1in}
% UC Merced:
% - 2010: ASME (Cal Poly SLO)
% - 2011: ESW (University at Buffalo)
% - 2013: ASME (San Diego)
%


%-----------------------------------------------------------------------------
% MENTORING
%-----------------------------------------------------------------------------
\section{Student Mentorship}
\vspace{0.1in}

% - undergrads (UCSD):
%   - Jeremy Orsoco: now PhD at UCSD
%   - Alex Corliss: now at SpaceX
%   - Marina Fernandez: now at
%   - Jocelyn Lu: now at
%   - Jessica Mart: now MS at UCSD
%   - Renn Darawali: now MS at UCSD
%   - Lorenzo Page
%   - Ciara Dooley: now at Northrop Grumman
%   - Stuart Sapia
\textbf{Undergraduate Research}, University of California, San Diego \\
Jeremy Orosco \hfill 2012--2014 \\
Alex Corliss \hfill 2012--2013 \\
Marina Fernandez \hfill 2012--2014 \\
Jocelyn Lu \hfill 2013--2014 \\
Jessica Mart \hfill 2014--2015 \\
Renn Darawali \hfill 2014--2015 \\
Lorenzo Page \hfill 2013--present \\
Ciara Dooley \hfill 2013 \\
Stuart Sapia \hfill 2015--present

% - senior design:
%   - MAE 126B (Senior Environmental Engineering Design Project): Jocelyn
\textbf{Senior Environmental Design Project} \hfill Spring 2014 \\
\textit{University of California, San Diego} \\
Project title: ``Self-Powered Weather Station for Environmental Research'' \\
Project members: Atalie Dajani, Kingston Hon, Leighann Huang, Jocelyn Lu

% - IDEA scholars:
%   - summer 2014: Lorenzo, Joanne, Christopher, etc.
%   - NASA projects:
%

% - visiting undergrads:
%   - Khari Rockward (Moorehouse College): STARS
%   - Jonathan Perez (Harvey Mudd)
%   - Mark Lozano (Pomona College)
%   - Jessica M (UFMG)
\textbf{Visiting Undergraduates}, University of California, San Diego \\
Khari Rockward (Moorehouse College) \hfill Summer 2012, 2013 \\
Jonathan Perez (Harvey Mudd) \hfill Summer 2014 \\
Mark Lozano (Pomona College) \hfill Summer 2015

% - HS students: low-cost solar tracker
%   - Leah Harvey
%   - Madeline Song
%   - Miya Coimbra
%   - Varkey Alumootil
\textbf{High School Students}, University of California, San Diego \\
Leah Harvey, Madeline Song, Miya Coimbra, Varkey Alumootil \hfill Summer 2015


%-----------------------------------------------------------------------------
% OUTREACH
%-----------------------------------------------------------------------------
\section{Outreach and Community Service}
\vspace{0.1in}
% - CER: High Tech Fair, school science fair
% - IDEA: Triton Day, grad student panels, faculty-student mixers (freshmen, transfers, current students)
% - UCSD: Graduate School Diversity Forum
% - other: Triton Summer STEM Academy program (workshop)
% - technical workshops: Python, CAD/Solidworks, Matlab
% - service: contributor to open source software (Homebrew)
%

\textbf{Center for Energy Research: Outreach Council}, Volunteer \hfill 2014--present \\
\textit{University of California, San Diego} \\
Presented solar energy demonstrations at events in the San Diego area.

\textbf{SWEET Workshop Series}, IDEA Student Center \hfill 2015--present \\
\textit{University of California, San Diego} \\
Co-developed a set of technical workshops for undergraduate engineers. \\
Taught workshops on Python, Matlab, Solidworks, numerical methods, time-series
analysis and image processing.


%-----------------------------------------------------------------------------
% PROFESSIONAL ACTIVITIES
%-----------------------------------------------------------------------------
\section{Professional Activities}
\vspace{0.1in}

\textbf{Paper Reviewing} \\
Solar Energy, Renewable Energy, AMS Journal of Applied Meteorology and
Climatology


%-----------------------------------------------------------------------------
% HONORS AND AWARDS
%-----------------------------------------------------------------------------
% - Innovate to Grow Competition
% - CITRIS
% - ESW
% - ASME conference (Cal Poly SLO)
%
\section{Awards}
\vspace{0.1in}

\textbf{1st Place: People's Choice}, Innovate to Grow Competition \hfill Spring 2012 \\
\textit{University of California, Merced} \\
Project title: ``Microturbine for UC Merced Irrigation Canals'' \\
Team members: David Larson, Daniel Leong, Samuel Isaiah, Steven Fleming

\textbf{Distributed Power Generation Project} \hfill 2011 \\
Project title: Solar Powered Cargo Ship \\
Sponsors: ESW, SunEdison/MEMC, Autodesk \\
Award amount: \$8150

\textbf{Honorable Mention}, CITRIS Big Idea Competition \hfill Spring 2010 \\
Project title: ``Distributed Computing for Open Access Solar Forecasting'' \\
Team members: Ricardo Marquez, David Larson, Hugo Pedro \\
Award amount: \$1000

%\textbf{5th Place}, ASME Old Guard Oral Presentation Competition \hfill Spring 2010
%Entry title: ``Distributed Computing for Open Access Solar Forecasting'' \\


%-----------------------------------------------------------------------------
% AFFILIATIONS
%-----------------------------------------------------------------------------
\section{Affiliations}
\vspace{0.1in}

\textbf{ASME}, Student Member \hfill 2009--present


%-----------------------------------------------------------------------------
% SKILLS
%-----------------------------------------------------------------------------
\section{Technical Skills}
\vspace{0.1in}

\textbf{Data Science:} machine learning, numerical optimization, data
visualization, statistical data analysis, image processing, time-series
analysis

\textbf{Software:} Python (NumPy, SciPy, Pandas, scikit-learn, iPython),
MATLAB, Mathematica, C, Go, Julia, SQL, shell scripting, Git, Microsoft Office,
LaTeX, Pro/ENGINEER, Solidworks, command-line tools (vim, ssh, etc.)

\textbf{Hardware:} Arduino, Beaglebone, Raspberry Pi, XBee, analog and digital
sensors, I$^2$C, SPI, UART, machining (mill, lathe, CNC)

\textbf{Platforms}: Mac OS X, Linux, Windows

\end{resume}
\end{document}
