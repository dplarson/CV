\documentclass[10pt]{res}

% ignore underfull warnings
\hbadness=10000

% increase text height to fit on 1-page
\setlength{\textheight}{9.5in}

\usepackage{hyperref}


\begin{document}

\name{David P. Larson}
\address{University of California San Diego \\ 9500 Gilman Drive \#0411 \\ La Jolla, CA 92093--0411}
\address{%
    \hfill \href{mailto:dplarson@ucsd.edu}{dplarson@ucsd.edu} \\
    \hfill \url{http://ieng6.ucsd.edu/~dplarson} \\
    \hfill \url{http://github.com/dplarson}
}


\begin{resume}


%-----------------------------------------------------------------------------
% EDUCATION
%-----------------------------------------------------------------------------
\section{Education}
\vspace{0.1in}

\textbf{University of California San Diego}, La Jolla, CA \hfill in progress \\
Ph.D., Mechanical Engineering \\
Advisor: Carlos F.~M. Coimbra \\

\vspace{-0.2in}

\textbf{University of California San Diego}, La Jolla, CA \hfill 2014 \\
M.S., Mechanical Engineering \\

\vspace{-0.2in}

\textbf{University of California Merced}, Merced, CA \hfill 2012 \\
B.S., Mechanical Engineering


%-----------------------------------------------------------------------------
% CITIZENSHIP
%-----------------------------------------------------------------------------
\hspace{-0.55in} \textbf{Citizenship}: United States


%-----------------------------------------------------------------------------
%
%-----------------------------------------------------------------------------
\section{Work Experience}
\vspace{0.1in}

\textbf{Electric Power Research Institute} \hfill Summer 2018 \\
\textit{Student Employee}, Power Delivery and Utilization \\
Analyzed distributed PV (DPV) and its impacts on load forecasting. \\
Developed a Python software package to generate, evaluate, and visualize DPV and load forecasts.

\textbf{University of California San Diego} \hfill 2012--2018 \\
\textit{Graduate Student Researcher}, Coimbra Research Group \\
Developed solar forecasting methods for large-scale solar plants. \\
Maintained research instruments and computing infrastructure (servers, databases, etc.).

\textbf{University of California Merced} \hfill 2009--2012 \\
\textit{Undergraduate Student Researcher}, Coimbra Research Group \\
Maintained research instruments used by multiple, funded solar forecasting projects.


%-----------------------------------------------------------------------------
% RESEARCH
%-----------------------------------------------------------------------------
\section{Research Experience}
\vspace{0.1in}

\textbf{University of California San Diego}, with Carlos F.~M. Coimbra \hfill 2012--present\\
Developed solar power forecasting methods for large-scale photovoltaic power plants. \\
Designed low-cost, solar-powered wireless sensor networks.

\textbf{University of California Berkeley}, with Robert Dudley \hfill Summer 2011 \\
\textit{Visiting UC LEADS Scholar}, Cal NERDS Program \\
Evaluated the effects of turbulence on hummingbird flight dynamics.

\textbf{University of California Merced}, with Carlos F.~M. Coimbra \hfill Summer 2010 \\
\textit{UC LEADS Scholar} \\
Assisted in the development of an experiment for studying insect flight aeroelastics.


%-----------------------------------------------------------------------------
% WORK
%-----------------------------------------------------------------------------
%
% UC Merced:
% - PIER RESCO
% - UC LEADS
%
% UC San Diego
% - CEC EPC-14-008
% - NSF and DOE SUNSHOT: included some WSN work
% - DOE

%\section{Work Experience}
%\vspace{0.1in}
%% UC Merced: PIER-RESCO project
%% UCSD: GSR, TA
%%
%
%\textbf{UC Merced}, Undergraduate Researcher \hfill 2009--2012 \\
%PIER RESCO Solar Community at UC Merced


%-----------------------------------------------------------------------------
% PUBLICATIONS
%-----------------------------------------------------------------------------
\section{Journal Publications}
\vspace{0.1in}

%\textbf{D.~P. Larson} and C.~F.~M. Coimbra (2018) ``Collaborative learning of distributed solar power forecasting'', (in preparation).

%\textbf{D.~P. Larson} and C.~F.~M. Coimbra (2018) ``On the use of spatially distributed telemetry for intra-hour solar power forecasting of utility-scale photovoltaic power plants'', (in preparation).

\textbf{D.~P. Larson} and C.~F.~M. Coimbra (2018) ``Direct power output forecasts from remote sensing image processing'', \textit{Journal of Solar Energy Engineering} 140(2), 02111. doi: \href{http://dx.doi.org/10.1115/1.4038983}{10.1115/1.4038983}

\textbf{D.~P. Larson}, L. Nonnenmacher and C.~F.~M. Coimbra (2016) ``Day-ahead forecasting of solar power output from photovoltaic plants in the American Southwest'', \textit{Renewable Energy} (91), pp. 11--20. doi: \href{http://dx.doi.org/10.1016/j.renene.2016.01.039}{10.1016/j.renene.2016.01.039}


%-----------------------------------------------------------------------------
%
%-----------------------------------------------------------------------------
\section{Technical Reports}
\vspace{0.1in}

R. Ferrera, T. Marvin, \textbf{D.~P. Larson}, T. Lindsay, and D. Falk (2017) ``Battery Energy Storage in Florida: Value, Challenges, and Opportunities''. \url{http://gps.ucsd.edu/_files/research/battery_energy_storage_in_florida.pdf}



%-----------------------------------------------------------------------------
% EMPLOYMENT HISTORY
%-----------------------------------------------------------------------------
%\section{Employment History}

% CEC
% ENG 10: Associate-In
% ENG 1-3: TA
% Graduate Student Researcher


%-----------------------------------------------------------------------------
% TEACHING
%-----------------------------------------------------------------------------
%\vspace{-0.1in}
\section{Teaching Experience}
\vspace{0.1in}

\textbf{Instructor}, ENG 10: Fundaments of Engineering Applications \hfill Winter 2018 \\
\textbf{Instructor}, ENG 10: Fundaments of Engineering Applications \hfill Fall 2017 \\
\textbf{Instructor}, ENG 10: Fundaments of Engineering Applications \hfill Spring 2017 \\
\textbf{Instructor}, ENG 10: Fundaments of Engineering Applications \hfill Winter 2017 \\
\textbf{Instructor}, ENG 10: Fundaments of Engineering Applications \hfill Fall 2016 \\
\textit{University of California San Diego} \\
Lectured on engineering mathematics, applications, and design.

\textbf{Teaching Assistant}, ENG 3: Orientation to Engineering III \hfill Spring 2016 \\
\textbf{Teaching Assistant}, ENG 3: Orientation to Engineering III \hfill Spring 2015 \\
\textbf{Teaching Assistant}, ENG 3: Orientation to Engineering III \hfill Spring 2014 \\
\textit{University of California San Diego} \\
Lectured on project management, engineering as a profession, and engineering ethics.

\textbf{Teaching Assistant}, ENG 2: Orientation to Engineering II \hfill Winter 2016 \\
\textbf{Teaching Assistant}, ENG 2: Orientation to Engineering II \hfill Winter 2015 \\
\textbf{Teaching Assistant}, ENG 2: Orientation to Engineering II \hfill Winter 2014 \\
\textit{University of California San Diego} \\
Lectured on career planning, professionalism, resume development, and presentation skills.

\textbf{Teaching Assistant}, ENG 1: Orientation to Engineering II \hfill Fall 2015 \\
\textbf{Teaching Assistant}, ENG 1: Orientation to Engineering II \hfill Fall 2014 \\
\textbf{Teaching Assistant}, ENG 1: Orientation to Engineering II \hfill Fall 2013 \\
\textit{University of California San Diego} \\
Lectured on academic planning, time management, and study habits.



%-----------------------------------------------------------------------------
% MENTORING
%-----------------------------------------------------------------------------
\section{Research Mentoring}
\vspace{0.1in}

% - undergrads (UCSD):
%   - Jeremy Orsoco: now PhD at UCSD
%   - Alex Corliss: now at SpaceX
%   - Marina Fernandez: now at
%   - Jocelyn Lu: now at
%   - Jessica Mart: now MS at UCSD
%   - Renn Darawali: now MS at UCSD
%   - Lorenzo Page
%   - Ciara Dooley: now at Northrop Grumman
%   - Stuart Sapia
\textbf{Undergraduate students} \\
Jeremy Orosco (currently Ph.D. student at UC San Diego) \hfill 2012--2014 \\
Alex Corliss \hfill 2012--2013 \\
Marina Fernandez (UC LEADS program) \hfill 2012--2014 \\
Khari Rockward (UCSD STARS program) \hfill 2012--2014 \\
Ciara Dooley \hfill 2013 \\
Jocelyn Lu \hfill 2013--2014 \\
Jonathan Perez \hfill Summer 2014 \\
Jessica Mart \hfill 2014--2015 \\
Renn Darawali \hfill 2014--2015 \\
Lorenzo Page \hfill 2013--2016 \\
Stuart Sapia (currently M.S. student at UC Berkeley) \hfill 2015--2017 \\
Mark Lozano \hfill Summer 2015 \\
Jessica Medrado (currently Ph.D. student at UC San Diego) \hfill 2016 \\
Mai Nong \hfill 2016--2017 \\
Joshua Mumford \hfill 2017--2018


\textbf{High School students} \\
Leah Harvey \hfill Summer 2015 \\
Madeline Song \hfill Summer 2015 \\
Miya Coimbra \hfill Summer 2015 \\
Varkey Alumootil \hfill Summer 2015 \\
Bruce Markman (MAP program) \hfill Summer 2017 \\
Danial Beg \hfill Summer 2017 \\
Harris Beg \hfill Summer 2017 \\
Delara Aryan (MAP program) \hfill 2017--2018 \\
Daniel Pak (MAP program) \hfill 2017--2018 \\
Anthony Nguyen (MAP program) \hfill 2017--2018



%-----------------------------------------------------------------------------
% OUTREACH
%-----------------------------------------------------------------------------
\section{Outreach and Community Service}
\vspace{0.1in}
% - CER: High Tech Fair, school science fair
% - IDEA: Triton Day, grad student panels, faculty-student mixers (freshmen, transfers, current students)
% - UCSD: Graduate School Diversity Forum
% - other: Triton Summer STEM Academy program (workshop)
% - technical workshops: Python, CAD/Solidworks, Matlab
% - service: contributor to open source software (Homebrew)
%

\textbf{Center for Energy Research: Outreach Council}, Volunteer \hfill 2014--present \\
\textit{University of California San Diego} \\
Presented renewable energy demonstrations at events in the San Diego area that
target traditionally underrepresented student populations, including the San Diego
High Tech Fair and the Expanding Your Horizons (EYH) annual conference.

\textbf{Center for Energy Research: Diversity Committee}, Co-Chair \hfill 2016--present \\
\textit{University of California San Diego} \\
Helped coordinate diversity initiatives by members of the Center for Energy Research.

\textbf{SWEET Workshop Series}, IDEA Student Center \hfill 2015--present \\
\textit{University of California San Diego} \\
Co-developed a set of technical workshops for undergraduate engineers. \\
Taught workshops on programming (Python and Matlab), Data Science, Deep
Learning, CAD (Solidworks), numerical methods, time-series analysis, image
processing, machine learning, 3D printing, and microcontrollers (Arduino).

\textbf{Engineering Graduate \& Scholarly Talks}, IDEA Student Center \hfill 2016--present \\
\textit{University of California San Diego} \\
Taught workshops on programming (Python), Data Science, Deep Learning, and
technical document preparation (LaTeX).

\textbf{Student Panelist} \hfill 2013--present \\
\textit{University of California San Diego} \\
Served as a panelist for a range of events focused on graduate school,
undergraduate research, and first-generation and traditionally underrepresented
students.


%-----------------------------------------------------------------------------
% PROFESSIONAL ACTIVITIES
%-----------------------------------------------------------------------------
\section{Professional Activities}
\vspace{0.1in}

\textbf{Paper Reviewing} \\
Solar Energy, Renewable Energy, AMS Journal of Applied Meteorology and
Climatology, ASME Journal of Solar Energy Engineering, IEEE Transactions on
Industrial Informatics


%-----------------------------------------------------------------------------
% HONORS AND AWARDS
%-----------------------------------------------------------------------------
% - Innovate to Grow Competition
% - CITRIS
% - ESW
% - ASME conference (Cal Poly SLO)
%
%\section{Awards}
%\vspace{0.1in}
%
%\textbf{1st Place: People's Choice}, Innovate to Grow Competition \hfill Spring 2012 \\
%\textit{University of California Merced} \\
%Project title: ``Microturbine for UC Merced Irrigation Canals'' \\
%Team members: David Larson, Daniel Leong, Samuel Isaiah, Steven Fleming
%
%\textbf{Distributed Power Generation Project} \hfill 2011 \\
%Project title: Solar Powered Cargo Ship \\
%Sponsors: ESW, SunEdison/MEMC, Autodesk \\
%Award amount: \$8150
%
%\textbf{Honorable Mention}, CITRIS Big Idea Competition \hfill Spring 2010 \\
%Project title: ``Distributed Computing for Open Access Solar Forecasting'' \\
%Team members: Ricardo Marquez, David Larson, Hugo Pedro \\
%Award amount: \$1000
%
%\textbf{5th Place}, ASME Old Guard Oral Presentation Competition \hfill Spring 2010
%Entry title: ``Distributed Computing for Open Access Solar Forecasting'' \\


%-----------------------------------------------------------------------------
% AFFILIATIONS
%-----------------------------------------------------------------------------
\section{Affiliations}
\vspace{0.1in}

\textbf{Bouchet Graduate Honor Society}, Graduate Student Member \hfill 2018--present \\
\textbf{American Society of Mechanical Engineers (ASME)}, Student Member \hfill 2009--present \\
\textbf{Society of Industrial Applied Mathematics (SIAM)}, Student Member \hfill 2017--present \\
\textbf{UC LEADS}, Scholar and Alumni \hfill 2010--present


%-----------------------------------------------------------------------------
% SKILLS
%-----------------------------------------------------------------------------
\section{Technical Skills}
\vspace{0.1in}

\textbf{Data Science:} machine learning, numerical optimization, convex
optimization, data visualization, statistical data analysis, image processing,
time-series analysis and forecasting

\textbf{Software:} Python (NumPy, SciPy, Pandas, scikit-learn, Jupyter),
MATLAB, Mathematica, C, Go, Julia, MySQL, shell scripting, Git, Microsoft Office
(Word, Excel, Powerpoint), LaTeX, Solidworks, Pro/ENGINEER, command-line tools
(vim, ssh, etc.), PyTorch, TensorFlow, Keras

\textbf{Hardware:} Arduino, Beaglebone, Raspberry Pi, XBee/ZigBee, analog and
digital sensors, I$^2$C, SPI, UART, machining (mill, lathe, CNC), rapid
prototyping (3D printing, lasercamm)

\textbf{Platforms}: Mac, Linux, Windows

\end{resume}
\end{document}
